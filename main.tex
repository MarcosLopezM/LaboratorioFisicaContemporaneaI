\documentclass[12pt]{IEEEtran}
\IEEEoverridecommandlockouts
\usepackage{fancyhdr}
\usepackage{graphicx}
\usepackage[spanish, es-tabla, es-nodecimaldot]{babel}
% \usepackage[utf8]{inputenc}
\usepackage{csquotes}
\usepackage{wrapfig}
\usepackage[l3]{csvsimple}
\usepackage{array}
\usepackage[calc, spanish]{datetime2}
\usepackage{enumitem}
% \usepackage{multicol}
\usepackage{chemformula}
\usepackage{multirow}
\usepackage{mismath}
\usepackage{adjustbox}
\usepackage{nccmath}
\usepackage{amsmath}
\usepackage{amssymb}
\usepackage{mathtools}
\usepackage{amsfonts,latexsym} % para tener disponibilidad de diversos simbolos
\usepackage{enumerate}
\usepackage{empheq}
\usepackage{derivative}
\usepackage{float}
\usepackage{threeparttable}
\usepackage{ifpdf}
\usepackage{rotating}
\usepackage{stfloats}
\usepackage{tabularray}
\usepackage{url}
% \usepackage[inline]{showlabels}
\usepackage{kantlipsum}
\usepackage{siunitx}
\usepackage{makecell}%To keep spacing of text in tables
\setcellgapes{2pt}%parameter for the spacing in tables
\usepackage{afterpage}
\usepackage[
  sorting=none,
  backend=biber,
  style=ieee,
  % bibstyle = authoryear,
  citestyle=numeric-comp,
]{biblatex}
\usepackage{hyperref}
\usepackage{cleveref}
\crefname{table}{tabla}{tablas} % Table's cross-reference name
\crefname{equation}{ec.}{ecs.} %
\newcommand\crefrangeconjunction{--}
\newcommand\crefpairconjunction{~y~}
\providecommand{\abs}[1]{\lvert#1\rvert}
%%%%%%%%%%%%%%%%%%%%%%%%%%%%%%%%%%%%%%%%%%%
%%% CREAR Y REESCRIBIR ALGUNOS COMANDOS %%%
%%%%%%%%%%%%%%%%%%%%%%%%%%%%%%%%%%%%%%%%%%%
\newcolumntype{P}[1]{>{\centering\arraybackslash}p{#1}}  %% Se crea un nuevo tipo de columna llamada P.
\newcommand{\tabitem}{~~\llap{\textbullet}~~}
\newcommand{\ctt}{\centering\scriptsize\textbf} %%\ctt abrevia el comando \centering\scriptsize\textbf
\newcommand{\dtt}{\scriptsize\textbf} %%\dtt abrevia el comando \scriptsize\textbf
\renewcommand\IEEEkeywordsname{Palabras clave}

%% Crea una lista en dos columnas
\SetEnumitemKey{twocol}{%
  itemsep = 1\itemsep,
  parsep = 0.5\parsep,
  before = \raggedcolumns
  \begin{multicols}{2},
    after =
\end{multicols}}
%%%%%%%%%%%%%%%%%%%%%%%%%%%%%%%%%%%%%%%%%%%

% correct bad hyphenation here
\hyphenation{op-tical net-works semi-conduc-tor} %% Con este comando se especifican como pueden seprarse las sílabas adecuadamente en caso una palabra quede en dos lineas diferentes de texto

\graphicspath{ {figs/} {logos/}}  %%Ruta donde se encuentran las imágenes, que esté vacio indica que las imagenes están dentro de la misma carpeta que contiene el archivo .tex

\sisetup{
  output-decimal-marker = {.},
  uncertainty-mode = separate,
}
% adjust as needed
\addtolength{\footskip}{0\baselineskip}
\addtolength{\textheight}{-1\baselineskip}

%Paquete tikz para hacer diagramas y figuras
\usepackage{tikz}
\usetikzlibrary{arrows}
%\usepackage[spanish,es-noquoting]{babel}

%%%%%%%%%%%%%%%%%%%%%%%%%%%%%%%%
%%%%% INICIO DEL DOCUMENTO %%%%%
%%%%%%%%%%%%%%%%%%%%%%%%%%%%%%%%

\addbibresource{CaracterizacionPeliculaDelgadaAluminioHiPIMSTribologia.bib}

\DTMsavedate{duedate}{2024-03-25}% Año-Mes-Día -> Fecha de entrega
\DTMnewdatestyle{usvardate}{%
  \renewcommand{\DTMdisplaydate}[4]{%
    \DTMMonthname{##2}\nobreakspace% Mes
    \number##1% Año
  }%
  \renewcommand{\DTMDisplaydate}{\DTMdisplaydate}%
}

\DeclareSIUnit{\min}{min}

\begin{document}
%%%%%%%%%%%%%%%%%%%%%%%%%%%%
%%% TÍTULO DEL DOCUMENTO %%%
%%%%%%%%%%%%%%%%%%%%%%%%%%%%

\title{Análisis de plasma de nitruro de boro (\ch{BN}) por espectrometría de emisión óptica obtenido por RF Magnetron Sputtering}

%%%%%%%%%%%%%%%%%%%%%%%%%%%
%%%%%%%%% AUTORES %%%%%%%%%
%%%%%%%%%%%%%%%%%%%%%%%%%%%
\author{\IEEEauthorblockN{Jesús Diego Gómez Garnica, Marcos López Merino}\\
	\IEEEauthorblockA{\textbf{Profesor}: Julio César Cruz Cárdenas}\\
	\IEEEauthorblockN{\DTMusedate{duedate}}}

%%%%%%%%%%%%%%%%%%%%%%%%%%%
\twocolumn[
	\begin{@twocolumnfalse}
		\maketitle
		\begin{abstract}
			En este trabajo se estudió la composicion del plasma emitido por un deposito de boro por el metodo RF (magnetron sputtering), analizando el espectro de emision a una altura de 5 cm por encima del blanco y en el centro del mismo, por intervalos de tiempo para concretar una base de datos para su posterior analisis, donde estudiando estas franjas de emision, se obtuvieron elementos tales como Nitrogeno, Argon, Boro, Oxigeno, presenten en el deposito de la camara.

		\end{abstract}

		\begin{IEEEkeywords}
			HiPIMS, SEM, Scratch Test
		\end{IEEEkeywords}
	\end{@twocolumnfalse}
	\vspace{1cm}
]

%https://doi.org/10.1016/j.surfcoat.2022.128409 -> Aplicaciones

%%%%%%%%%%%%%%%%%%%%%%
%\IEEEpeerreviewmaketitle
%%%%%%%%%%%%%%%%%%%%%%%%%%%%%%%%%%%%%
%%% PRIMERA SECCIÓN DEL DOCUMENTO %%%
%%%%%%%%%%%%%%%%%%%%%%%%%%%%%%%%%%%%%
\section{Introducción}

Las películas delgadas son capas de material con espesores que van desde unos cuantos nanómetros hasta varios micrómetros y pueden encontrarse en revestimientos antirreflejantes para anteojos y hasta en chips de computadoras. Su ventaja radica en la combinación de las propiedades superficiales de la película delgada con las propiedades del material subyacente, lo que mejora las características químicas, mecánicas o eléctricas, entre otras.

Existen múltiples técnicas para producir películas delgadas, tales como electrodeposición (\emph{electroplating}), la evaporación, la pulverización catódica (\emph{sputter deposition}), la deposición química de vapor  (CVD, por sus siglas en inglés) y combinaciones de estas. En este trabajo nos enfocaremos en una técnica recientemente desarrollada llamada \emph{high-power impulse magnetron sputtering} (HiPIMS) \cite{lundinHiPIMSProcess2010}.

Sputtering RF: La pulverización catódica por radio frecuencia (RF) ocurre a frecuencias superiores a
50 kHz. En ésta, los iones no alcanzan suficiente movilidad como para establecer una descarga similar
a la del sputtering DC y los electrones tienen suficiente energía como para causar colisiones ionizantes
en el espacio entre los electrodos, lo que produce el plasma en el espacio entre los electrodos. Una de
las mayores ventajas de usar sputtering RF es que se pueden utilizar blancos no conductores pues se
aplica un potencial oscilante en el blanco, lo que hace que en cada medio ciclo se puedan acelerar los
iones del plasma hacia la superficie con suficiente energía como para producir la pulverización catódica
y en el otro medio ciclo, los electrones del plasma alcanzan la superficie impidiendo cualquier tipo de
acumulación de carga. Las frecuencias RF usadas para la deposición por sputtering están en el rango de
0.5 y 30 MHz siendo 13.56 MHz la frecuencia más usada comercialmente. Una de las mayores
desventajas de la pulverización catódica de materiales semiconductores o aislantes es que la mayoría
de estos materiales tienen baja conductividad térmica, grandes coeficientes de expansión térmica y son
usualmente materiales frágiles, y este tipo de propiedades son indeseables en un proceso como el RF
sputtering donde la mayoría de la energía de bombardeo produce calor y se generan grandes gradientes
de temperatura en el blanco, lo que puede producir su fractura

\section{Procedimiento experimental}

Mediante pruebas tribológicas se caracterizaron las películas delgadas de \ch{Al} depositadas en sustratos de \ch{Si} y portaobjetos de vidrio por la técnica de HiPIMS.

\subsection{Medición del espesor}

Para la medición del espesor de las películas se requirió colocar una gota de pegamento en alguno de los sustratos, en particular, el portaobjetos de vidrio. Posteriormente, se retiró el pegamento y mediante la técnica de perfilometría óptica se midió el espesor en distintas áreas del cráter generado alrededor de ésta. El perfilómetro óptico utilizado fue el Nexview de la marca ZYGO.



\subsection{Prueba de \emph{scratch}}

La prueba de \emph{scratch} se realizó usando el tester mecánico NANOVEA CB500 en la película depositada en el portaobjetos de vidrio siguiendo la norma ASTM-1624. Esta se prueba se llevó acabo a una temperatura \qty{22.7}{\degreeCelsius} utilizando un contracuerpo de hierro con cromo de \qty{1}{\mm} aplicando una carga progresiva de \qty{0.01}{\N} hasta \qty{2.5}{\N} y de \qty{0.01}{\N} hasta \qty{1}{\N} con una velocidad constante de \qty{1}{\mm\per\min} en una distancia de \qty{5}{\mm}.

\subsection{SEM}

Para la obtención de las imágenes SEM de la sección transversal y la superficie de la película depositaa en \ch{Si} se usó el microscopio JEOL JSM-6010LA, a una resolucion de 20,000 x y 10,000 x respectivamente.

\section{Resultados y análisis}

El grosor de la película al ser medida desde el perfilómetro óptico resulto no ser constante, este variaba significativamente en un promedio entre $250 nm$ a $300 nm$, esto dificulto tener una fiablilidad al realizar la prueba de scratch.

Al ver la pelicula con una resolucion mayor por medio de un microscopio la pelicula presento zonas irreguales donde no habia recubrimiento, sin embargo se procedio con la prueba, la fractura expuesta fue del tipo adhesiva, el material se abrio sobre la pelicula y parte del mismo quedo sobre el contracuerpo de 1mm de diametro, utilizamos este mismo contracuerpo para realizar la segunda prueba de rayado, en este caso antes de romperse y adherirse al contracuepro este dio saltos sobre la superfie, dejando el coeficiente de friccion con picos significativos

\section{Conclusiones}
\kant[5]


%%%%%%%%%%%%%%%%%%%%%%%%%%%%%%%%
%%%%%%    Bibliografia   %%%%%%%
%%%%%%%%%%%%%%%%%%%%%%%%%%%%%%%%
% \newpage
\nocite{*}
\printbibliography
%\section{Apéndices}
%\appendices

\end{document}
%%%%%%%%%%%%%%%%%%%%%%%%%%%%%%%%
%%%%%% FIN DEL DOCUMENTO %%%%%%%
%%%%%%%%%%%%%%%%%%%%%%%%%%%%%%%%
